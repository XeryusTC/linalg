\section{Matrix operaties (\S2.1)}
In een matrix $A$ staat $a_{ij}$ voor het getal in de $i^e$ rij en de $j^e$ kolom. Dus $a_{32}$ staat voor het getal in de $3^e$ rij en de $2^e$ kolom.
\subsection{Optellen}
Optellen gaat componentsgewijs.
\[ A = \begin{pmatrix*}[r] 1 & 2 \\ 3 & 4 \end{pmatrix*}, B = \begin{pmatrix*}[r]0 & -1 \\ 2 & 7 \end{pmatrix*}, C = \begin{pmatrix*}[r] 4 & 3 & 1 \\ -1 & 2 & 3 \end{pmatrix*} \]
\[ A+B = \begin{pmatrix*}[c] 1+0 & 2+ -1 \\
3+2 & 4+7 \end{pmatrix*} = \begin{pmatrix*}[r] 1 & 1 \\ 5 & 11 \end{pmatrix*} \]
$A+C$ kan niet omdat ze niet even groot zijn.

\paragraph{Bewijs}
\begin{eqnarray*}
	A \vec{x} = \begin{pmatrix*}[r] 1 & 2 \\ 3 & 4 \end{pmatrix*} \begin{vect} x \\ y \end{vect} &=& \begin{vect} x + 2y \\ 3x+4y \end{vect} \\
	B \vec{x} = \begin{pmatrix*}[r] 0 & -1 \\ 2 & 7 \end{pmatrix*} \begin{vect} x \\ y \end{vect} &=& \begin{vect} -y \\ 2x+7y \end{vect} \\
	\hline \\
	\left(\!\begin{pmatrix*}[r] 1 & 2 \\ 3 & 4 \end{pmatrix*} + \begin{pmatrix*}[r] 0 & -1 \\ 2 & 7 \end{pmatrix*}\!\right) &=& \begin{vect} x+y \\ 5x+11y \end{vect} \\
	\begin{pmatrix*}[r] 1 & 1 \\ 5 & 11 \end{pmatrix*} &=& \begin{pmatrix*}[r] 1 & 1 \\ 5 & 11 \end{pmatrix*} \begin{vect} x \\ y \end{vect}
\end{eqnarray*}

\subsection{Vermenigvuldigen}
Vermenigvuldigen gaat componentsgewijs.
\[2 \cdot C = 2 \cdot \begin{pmatrix*}[r] 4 & 3 & 1 \\ -1 & 2 & 3 \end{pmatrix*} = \begin{pmatrix*}[r] 8 & 6 & 2 \\ -2 & 4 & 6 \end{pmatrix*} \]

\subsection{Rekenregels}
\[\begin{array}{ll}
A+B = B+A & \quad \forall A, B \\
k(A+B) = kA+kB & \quad \forall k, A, B
\end{array} \]

\subsection{Formele definities}
\[\begin{array}{lll}

A = (a_{ij}) & i=1, \ldots, m ; j = 1, \ldots, n & \mbox{i rijen, j kolommen} \\
B = (b_{ij}) && \\
C = A+B & c_{ij} = a_{ij}+b_{ij} & \\
D = k \cdot A & d_{ij} = k \cdot a_{ij} \\
\sum\limits_{i=1}^n \rho_i = \rho_1 + \rho_2 + \ldots + \rho_{n-1} + \rho_n && \\
A = (a_{ij}) & i = 1, \ldots, m; j = 1, ldots, n & (m \times n) \\
B = (b_{ij}) & i = 1, \ldots, n; j = 1, ldots, p & (n \times p) \\
A \cdot B = C = (c_{ij}) & i=1, \ldots m; j = 1, \ldots, p & (m \times p)
\end{array} \]
$(m \times n) \times (n \times p) = (m \times p)$, het aantal kolommen in $A$ moet gelijk zijn aan het aantal rijen in $B$. Het inproduct van de $i^e$ rij van $A$ met de $j^e$ kolom van $B$:
\[ c_{ij} = \sum\limits_{l=1}^n a_{il} \cdot b_{lj} \]

\paragraph{Voorbeeld}
\[A = \begin{pmatrix*}[r] 1 & 2 \\ 3 & 4 \end{pmatrix*}, B = \begin{pmatrix*}[r] -1 & 2 \\ 4 & 7 \end{pmatrix*} \]
\[ C = A \cdot B = \begin{pmatrix*}[r] 1 & 2 \\ 3 & 4 \end{pmatrix*} \begin{pmatrix*}[r] -1 & 2 \\ 4 & 7 \end{pmatrix*} = \begin{pmatrix} -1 + 8 & 2+14 \\
-3 + 16 & 6 +28 \end{pmatrix} = \begin{pmatrix*}[r]7 & 16 \\ 13 & 34 \end{pmatrix*} \]

\paragraph{Voorbeeld}
\[ \left.\begin{array}{l} A = (3 \times 2) \\
B = (2 \times 2) \end{array}\right\} C = (3 \times 2) \]
\[ c_{ij} = \sum\limits_{l=1}^2 a_{il} \cdot b_{lj} = a_{i1} b_{1j} + a_{i2} b_{2j} \]

$a_{i1}$ en $a_{i2}$ zijn de componenten op de $i^e$ rij van $A$. $b_{1j}$ en $b_{2j}$ zijn de componenten van de $j^e$ kolom van $B$.

\paragraph{Voorbeeld}
\[ A = \begin{pmatrix*}[r] 4 & 3 \\ 7 & 2 \\ 9 & 0 \end{pmatrix*}, B = \begin{pmatrix*}[r] 2 & 5 \\ 1 & 6 \end{pmatrix*} \]
\[ C = A \cdot B = \begin{pmatrix*}[r] 4 & 3 \\ 7 & 2 \\ 9 & 0 \end{pmatrix*} \begin{pmatrix*}[r] 2 & 5 \\ 1 & 6 \end{pmatrix*} = \begin{pmatrix} 8+3 & 20+18 \\ 14+2 & 35+12 \\ 18+0 & 45+0 \end{pmatrix} = \begin{pmatrix*}[r] 11 & 38 \\ 16 & 47 \\ 18 & 45 \end{pmatrix*} \]

\paragraph{Voorbeeld}
\[ A = \begin{pmatrix*}[r] 5 & 1 \\ 3 & -2 \end{pmatrix*}, B = \begin{pmatrix*}[r] 2 & 0 \\ 4 & 3 \end{pmatrix*} \]
\[ A \cdot B = \begin{pmatrix*}[r] 5 & 1 \\ 3 & -2 \end{pmatrix*} \begin{pmatrix*}[r] 2 & 0 \\ 4 & 3 \end{pmatrix*} = \begin{pmatrix} 10+4 & 0+3 \\ 6-8 & 0-6 \end{pmatrix} = \begin{pmatrix*}[r] 14 & 3 \\ -2 & -6 \end{pmatrix*} \]
\[ B \cdot A = \begin{pmatrix*}[r] 2 & 0 \\ 4 & 3 \end{pmatrix*} \begin{pmatrix*}[r] 5 & 1 \\ 3 & -2 \end{pmatrix*} = \begin{pmatrix} 10+0 & 2+0 \\ 20+9 & 4-6 \end{pmatrix} = \begin{pmatrix*}[r] 10 & 2 \\ 29 & -2 \end{pmatrix*} \]

Let op: $A \cdot B \neq B \cdot A$. \index{$A \cdot B \neq B \cdot A$}

\paragraph{Voorbeeld}
\[ \begin{pmatrix} 1 & 1 \\ 100 & 100 \end{pmatrix} \begin{pmatrix*}[r] -1 & 1 \\ 1 & -1 \end{pmatrix*} = \begin{pmatrix*}[r] 0 & 0 \\ 0 & 0 \end{pmatrix*} \]

\textbf{Waarschuwing:} dit betekent dus dat als $AB = 0$ dat niet betekent dat $A=0$ of $B=0$.

\[ \begin{pmatrix*}[r] -1 & 1 \\ 1 & -1 \end{pmatrix*} \begin{pmatrix} 1 & 1 \\ 100 & 100 \end{pmatrix} = \begin{pmatrix*}[r] 99 & 99 \\ -99 & -99 \end{pmatrix*} \neq 0 \]

\textbf{Waarschuwing:} $AB=0$ betekent dus niet dat $BA=0$

\[ \begin{pmatrix} 1 & 1 \\ 100 & 100 \end{pmatrix} \begin{pmatrix*}[r] -3 & 1 \\ 3 & -1 \end{pmatrix*} = \begin{pmatrix*}[r] 0 & 0 \\ 0 & 0 \end{pmatrix*} \]

\textbf{Waarschuwing:} $AB = AC$ betekent niet dat $B=C$.

\subsection{Rekenregels}
\[ A(B+C) = AB + AC \]
\[ (AB)C = A(BC) \]
\[ (B+C)A = BA + CA \]

\paragraph{Twee praktijd voorbeelden}
\begin{enumerate}
\item Verhuizen:
	\[ \vec{x_{n+1}} = A \vec{x_n} \quad \vec{x_0} = \mbox{start} \]
	\[ \vec{x_2} = A \vec{x_1} = AA \vec{x_0} \]
\item Economie:
\begin{table}[h!]
	\begin{tabular}{l|cc}
		Kosten per product & Product 1 &  Product 2 \\
		\hline
		Grondstof & 1,2 & 1,6 \\
		Arbeid & 0,3 & 0,4 \\
		Diversen & 0,5 & 0,6
	\end{tabular}
	\caption{Matrix A $(3 \times 2)$}
\end{table}

\begin{table}[h!]
	\begin{tabular}{l|cccc}
		Productie per kwartaal & Q1 & Q2 & Q3 & Q4 \\
		\hline
		Product 1 & 3 & 8 & 6 & 9 \\
		Product 2 & 6 & 2 & 4 & 3
	\end{tabular}
	\caption{Matrix B $(2 \times 4)$}
\end{table}

De kosten per kwartaal kun je berekenen door de tabellen te combineren als matrix.
\[ \frac{\mbox{kosten}}{\mbox{prod}} \times \frac{\mbox{prod}}{\mbox{kwartaal}} \to A \cdot B \]
\[ AB = \begin{pmatrix*}[r] 1,2 & 1,6 \\ 0,3 & 0,4 \\ 0,5 & 0,6 \end{pmatrix*} \begin{pmatrix*}[r] 3 & 8 & 6 & 9 \\ 6 & 2 & 4 & 3 \end{pmatrix*} = \begin{pmatrix*}[r]
	13,2 & 12,8 & 13,6 & 15,6 \\
	3,3 & 3,2 & 3,4 & 3,9 \\
	5,1 & 5,2 & 5,4 & 6,3
\end{pmatrix*} \]

Hierbij zijn de kolommen de kwartalen en elke rij is een type van de kosten.

\end{enumerate}

\subsection{Speciale matrices}
\subsubsection{Macht verheffen}
\[ A^k = A \cdot A \cdot ~ \cdots ~ \cdot A \quad \mbox{(k-keer)} \]

\[ A^0 = I = \begin{pmatrix}
	1 & & &\sigma \\
	& 1 & &\\
	& & \ddots & \\
	\sigma & & & 1
\end{pmatrix} \]
	
	Hierbij staat $I$ voor de \emph{eenheidsmatrix} \index{eenheidsmatrix}, de matrix met alleen maar \'e\'enen op de diagonaal, de rest zijn nullen (aangegeven met $\sigma$. Deze diagonaal noemen we de \emph{hoofddiagonaal}. \index{hoofddiagonaal} De hoofddiagonaal bij een $(n \times n)$ matrix bestaat altijd uit de elementen $a_{11}, a_{22}, \ldots, a_{nn}$.
	
	$AI = A$ en $IA = A$ dus $I$ is neutraal bij vermenigvuldiging.

\subsubsection{Getransponeerde matrix}
\[ A = (a_{ij}) \]
\[ A^T = (a_{ji}) \]
Dit heet ook wel de getransponeerde matrix, je verwisselt de rijen met de kolommen.

\paragraph{Voorbeeld}
\[ A = \begin{pmatrix} 1 & 2 \\ 3 & 4 \end{pmatrix} \to A^T = \begin{pmatrix} 1 & 3 \\ 2 & 4 \end{pmatrix} \]
Dit komt neer op spiegelen in de hoofddiagonaal.

\paragraph{Voorbeeld}
\[ B = \begin{pmatrix*}[r] 5 & -8 & 1 \\ 4 & 0 & 0 \end{pmatrix*} \to A^T = \begin{pmatrix*}[r] 5 & 4 \\ -8 & 0 \\ 1 & 0 \end{pmatrix*} \]
Hierbij ga je dus van $(2 \times 3)$ naar $(3 \times 2)$.

\paragraph{Voorbeeld}
\[C = \begin{pmatrix*} 6 & 2 & 3 \end{pmatrix*} \to C^T = \begin{vect} 6 \\ 2 \\ 3 \end{vect} \]
Gaat van $(1 \times 3)$ naar $(3 \times 1)$, oftewel een vector.

\subsubsection{Rekenregels voor $A^T$}
\begin{eqnarray*}
	{(A^T)}^T &=& A \\
	(A+B)^T &=& A^T + B^T \\
	(kA)^T &=& kA^T \\
	(AB)^T &=& B^T A^T \mbox{dus} (AB)^T \neq A^T B^T \\
\end{eqnarray*}

\subsubsection{Diagonaal matrix}
\[ D = \begin{pmatrix}
	d_1 & & &\sigma \\
	& d_2 & &\\
	& & \ddots & \\
	\sigma & & & d_m
\end{pmatrix} \]
\[ D^n = \begin{pmatrix}
	d_1^n & & &\sigma \\
	& d_2^n & &\\
	& & \ddots & \\
	\sigma & & & d_m^n
\end{pmatrix} \]
\[ I^N = I \]

\paragraph{Voorbeeld}
\[D = \begin{pmatrix*}[r] -1 & & \sigma \\ & 3 &  \\ \sigma && -2 \end{pmatrix*} \to D^2 = \begin{pmatrix*}[r] 1 & & \sigma \\ & 9 &  \\ \sigma && 4 \end{pmatrix*} \]