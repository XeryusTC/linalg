\section{$LU$ factorisatie (\S2.5)}
$A \vec{x} = \vec{b}$ waarbij $A$ zeer groot (denk in de miljoenen). $A = LU$ waarbij $L$ een beneden driehoeksmatrix is en $U$ een boven driehoeksmatrix.
\[ A = \begin{pmatrix*}[r]
	2 & 1 & -1 \\
	-2 & 2 & 0 \\
	-2 & 5 & 1
\end{pmatrix*} =
\begin{pmatrix*}[r]
	1 & & \sigma \\
	-1 & 1 &  \\
	-1 & 2 & 1 \end{pmatrix*}
\begin{pmatrix*}[r]
	2 & 1 & -1 \\
	& 3 & -1 \\
	\sigma & & 2
\end{pmatrix*} \]
$\vec{b} = \begin{vect} 3 \\ 2 \\ -3 \end{vect} \quad A \vec{x} = \vec{b}$ ofwel $LU \vec{x} = \vec{b} \to L \vec{y} = \vec{b}$
\begin{enumerate}
	\item $L \vec{y} = \vec{b} \to \vec{y}$
	\item $U \vec{x} = \vec{y}$
\end{enumerate}

\paragraph{Voorbeeld}
\[ L \vec{y} = \vec{b} \]
\[ \left(\!\begin{array}{rrr|r}
	1 & 0 & 0 & 3 \\
	-1 & 1 & 0 & 2 \\
	-1 & 2 & 1 & -3
\end{array}\!\right) \sim
\left(\!\begin{array}{rrr|r}
	1 & 0 & 0 & 3 \\
	0 & 1 & 0 & 5 \\
	0 & 0 & 1 & -10
\end{array}\!\right) \to \vec{y} = \begin{vect} 3 \\ 5 \\ -10 \end{vect} \]
\[ U \vec{x} = \vec{y} \]
\[ \left(\!\begin{array}{rrr|r}
	2 & 1 & -1 & 3 \\
	0 & 3 & -1 & 5 \\
	0 & 0 & 2 & -10
\end{array}\!\right) \sim
\left(\!\begin{array}{rrr|r}
	1 & 0 & 0 & 1 \\
	0 & 1 & 0 & 0 \\
	0 & 0 & 1 & -5
\end{array}\!\right) \to \vec{x} = \begin{vect} -1 \\ 0 \\ -5 \end{vect} \]

\section{Leontief (\S2.6)}
\[ \begin{vect} x_1 \\ x_2 \\ x_3 \end{vect} =
C \begin{vect} x_1 \\ x_2 \\ x_3 \end{vect} +
\begin{vect} d_1 \\ d_2 \\ d_3 \end{vect} \]
Hierbij is $\vec{x}$ de vector van producten, $C \vec{x}$ de consumptie en $\vec{d}$ is de export/demand.
\begin{eqnarray*}
	\vec{x} &=& C \vec{x} + \vec{d} \\
	I \vec{x} &=& C \vec{x} + \vec{d} \\
	(I-C) \vec{x} &=& \vec{d} \\
	\vec{x} &=& (I-C)^{-1} \vec{d}
\end{eqnarray*}

\section{Computer graphics (\S2.7)}
\subsection{Homogene co\"ordinaten}
Elk punt in $(x, y)$ in $\mathbb{R^2}$ kan als een nieuw punt $(x, y, 1)$ in $\mathbb{R^3}$ worden geschreven. Homogene co\"ordinaten worden niet be\"invloed door vermenigvuldigen met scalars maar wel door vermenigvuldiging met $(3 \times 3)$ matrices.

\paragraph{Voorbeeld} Lineaire transformaties bij homogene co\"ordinaten. De matrix bij de translatie ziet er nu uit als $\begin{pmatrix} A & 0 \\ 0 & 1 \end{pmatrix}$ waarbij $A$ een $(2 \times 2)$ matrix is.
\begin{eqnarray*}
	\mbox{Rotatie over de hoek } \varphi &\to& \begin{pmatrix*}[r]
		\cos \varphi & -\sin \varphi & 0 \\
		\sin \varphi & \cos \varphi & 0 \\
		0 & 0 & 1
	\end{pmatrix*} \\
	\mbox{Spiegeling door } y = x &\to& \begin{pmatrix}
		0 & 1 & 0 \\
		1 & 0 & 0 \\
		0 & 0 & 1
	\end{pmatrix} \\
	\mbox{Translatie van } (x, y) \mbox{ naar } (x+h,y+k) &\to& \begin{pmatrix}
		1 & 0 & h \\
		0 & 1 & k \\
		0 & 0 & 1
	\end{pmatrix}
\end{eqnarray*}

\subsection{Homogene 3D co\"ordinaten}
Hetzelfde principe geldt ook bij co\"ordinaten in 3 dimensies. $(x, y, z)$ in $\mathbb{R^3}$ wordt dan $(x, y, z, 1)$. In het algemeen zijn $(X, Y, Z, H)$ homogene co\"ordinaten voor $(x, y, z)$ als $H \neq 0$ en $x = \frac{X}{H}, y = \frac{Y}{H}, z = \frac{Z}{H}$.
\paragraph{Voorbeeld} Het co\"ordinaat $(5, -3, 7)$ kan ook uitgedrukt worden met behulp van een homogene co\"ordinaat vermenigvuldigd met een scalar. $(10, -6, 14, 2)$ en $(-15, 9, -21, -3)$ zijn dus synoniem aan $(5, -3, 7)$.