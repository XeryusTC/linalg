\section{Blokmatrices (\S2.4)}
\paragraph{Voorbeeld} 
\[ A = \left(\!\begin{array}{rr|r} 2 & 5 & 0 \\
-3 & -7 & 0 \\
\hline
0 & 0 & 2 \end{array}\!\right) \]

In het algemeen:
\[ A = \begin{pmatrix} A_{11} & A_{12} \\
A_{21} & A_{22} \end{pmatrix} \]
Hierbij mag $A_{ij}$ ook een getal zijn, het hoeft dus geen matrix te zijn.

\[ \vec{b} \]
\[ A \vec{b} = \left(\!\begin{array}{rr|r} 2 & 5 & 0 \\
-3 & -7 & 0 \\ \hline
0 & 0 & 2 \end{array}\!\right) \begin{vect} 1 \\ 2 \\ \hline 3 \end{vect} = \begin{vect} \begin{pmatrix*}[r] 2 & 5 \\ -3 & -7 \end{pmatrix*} \begin{vect} 1 \\ 2 \end{vect} + \begin{vect} 0 \\ 0 \end{vect} \cdot 3 \\
\begin{pmatrix} 0 & 0 \end{pmatrix} \begin{vect} 1 \\ 2 \end{vect} + 2 \cdot 3
\end{vect} = \begin{vect} 12 + 0 \\ -17 + 0 \\ 0 + 6 \end{vect} =
\begin{vect} 12 \\ -17 \\ 6 \end{vect} \]

\subsection{Diagonaal blokmatrix}
\[A = \begin{pmatrix} A_{11} & \sigma \\
\sigma & A_{22} \end{pmatrix}, A_{11} = \begin{pmatrix*}[r] 2 & 5 \\ -3 & -7 \end{pmatrix*}, A_{22} = 2 \]
\[ A^2 = \begin{pmatrix} A_{11}^2 & \sigma \\
\sigma & A_{22}^2 \end{pmatrix} = \begin{pmatrix} \begin{pmatrix*} 2 & 5 \\ -3 & -7 \end{pmatrix*}^2 & \sigma \\
\sigma & 4 \end{pmatrix} \]
\[ A^{-1} = \begin{pmatrix} A_{11}^{-1} & \sigma \\
\sigma & A_{22}^{-1} \end{pmatrix} = \left(\!\begin{array}{rr|r} 7 & -5 & 0 \\
3 & 2 & 0 \\ \hline
0 & 0 & \sfrac{1}{2} \end{array}\!\right), A_{11}^{-1} = \frac{1}{-14+15} \begin{pmatrix*}[r] 7 & -5 \\ 3 & 2 \end{pmatrix*} = \begin{pmatrix*} 7 & -5 \\ 3 & 2 \end{pmatrix*} \]

\paragraph{Voorbeeld} Geen diagonaal blokmatrix?
$A = \begin{pmatrix} A_{11} & A_{12} \\
A_{21} & A_{22} \end{pmatrix}$ Wat is $A^{-1}$? Zoek $B$ zodanig dat $AB=I$
\[\begin{pmatrix} A_{11} & A_{12} \\
\sigma & A_{22} \end{pmatrix} \begin{pmatrix} B_{11} & B_{22} \\
B_{21} & B_{22} \end{pmatrix} = \begin{pmatrix}
I & \sigma \\ \sigma & I \end{pmatrix} \]
Na uitvermenigvuldigen krijg je:
\begin{eqnarray*}
	A_{11}B_{11} + A_{12}B_{21} &=& I \\
	A_{11}B_{12} + A_{12}B_{22} &=& \sigma \\
	\sigma B_{11} + A_{22}B_{21} &=& A_{22}B_{21} = \sigma \\
	\sigma B_{12} + A_{22}B_{22} &=& A_{22}B_{22} = I
\end{eqnarray*}
Werk dit als volgt uit:
\begin{enumerate}
	\item \begin{eqnarray*}
		A_{22}B_{21} &=& \sigma \\
		A_{22}^{-1}A_{22}B_{21} &=& A_{22}^{-1} \sigma \\
		B_{21} &=& \sigma
	\end{eqnarray*}
	
	\item \begin{eqnarray*}
		A_{11}B_{11} + A_{12}B_{21} &=& I \\
		A_{11}B_{11} &=& I \\
		A_{11}^{-1}A_{11}B_{11} &=& A_{11}^{-1}I \\
		B_{11} &=& A_{11}^{-1}
	\end{eqnarray*}
	
	\item \begin{eqnarray*}
		A_{22}B_{22} &=& I \\
		A_{22}^{-1}A_{22}B_{22} &=& A_{22}^{-1}I \\
		B_{22} &=& A_{22}^{-1}
	\end{eqnarray*}
	
	\item \begin{eqnarray*}
		A_{11}B_{12} + A_{12}B_{22} &=& \sigma \\
		A_{11}B_{12} + A_{12}A_{22}^{-1} &=& \sigma \\
		A_{11}B_{12} &=& -A_{12}A_{22}^{-1} \\
		B_{12} &=& -A_{11}^{-1}A_{12}A_{22}^{-1}
	\end{eqnarray*}
\end{enumerate}
\[ B = \begin{pmatrix} A_{11}^{-1} & A_{11}^{-1}A_{12}A_{22}^{-1} \\
\sigma & A_{22}^{-1} \end{pmatrix} = A^{-1} \]
Het berekenen van $A^{-1}$ hoeft dus niet altijd makkelijker te zijn op deze manier!