\section{Lineaire transformaties (\S 1.8, \S 1.9)}
De matrix vergelijking $A \vec{x} = \vec{b}$ is equivalent aan de vector vergelijking $x_1 \vec{a}_1 + \cdots + x_n \vec{a}_n = \vec{b}$. Matrix $A$ werkt/acts op een vector $\vec{x}$ en maakt/produces vector $\vec{b} \to A$ transformeert $\vec{x}$ in $\vec{b} \leftrightarrow A$  beeldt $\vec{x}$ af op $\vec{b}$.

\paragraph{Voorbeeld}
\[ \begin{pmatrix*}[r]
	4 & -3 & 1 & 3 \\
	2 & 0 & 5 & 1
\end{pmatrix*} \begin{vect} 1 \\ 1 \\ 1 \\ 1 \end{vect} = \begin{vect} 5 \\ 8 \end{vect} \]
\[ \begin{pmatrix*}[r]
	4 & -3 & 1 & 3 \\
	2 & 0 & 5 & 1
\end{pmatrix*} \begin{vect} 1 \\ 4 \\ -1 \\ 3 \end{vect} = \begin{vect} 0 \\ 0 \end{vect} = \vec{0} \]

$A \vec{x} = \vec{b} \to$ zoek alle vectoren $\vec{x} \in \mathbb{R}^4$ die transformeren in $\vec{b} \in \mathbb{R}^2$ onder 'werking' van matrix $A$.

\subsection{Definitie transformatie}
Transformatie T \index{transformatie}: $\mathbb{R}^n \to \mathbb{R}^m$ kent aan iedere vector $\vec{x} \in \mathbb{R}^n$ een vector $T(\vec{x})$ in $\mathbb{R}^m$ toe volgens een $m \times n$ matrix $A$ ($m$ rijen, $n$ kolommen)

\paragraph{Voorbeeld} $T: \mathbb{R}^2 \to \mathbb{R}^3, A = \begin{pmatrix*}[r]
	1 & -3 \\
	3 & 5 \\
	-1 & 7
\end{pmatrix*}$
\begin{enumerate}
	\item Bereken $T(\vec{u})$ met $\vec{u} = \begin{vect} 2 \\ -1 \end{vect}$
	\[ T(\vec{u}) = A \vec{u} = \begin{pmatrix*}[r]
	1 & -3 \\
	3 & 5 \\
	-1 & 7
\end{pmatrix*} \begin{vect} 2 \\ -1 \end{vect} = \begin{vect} 5 \\ 1 \\ -9 \end{vect} \]
	\item Zoek vector $\vec{x} \in \mathbb{R}^2$ die onder $A$ transformeert naar $\vec{b} = \begin{vect} 3 \\ 2 \\ -5 \end{vect}$. $T(\vec{x}) = \vec{b}$ ofwel $A \vec{x} = \vec{b}$
	\[ \begin{pmatrix*}[r]
	1 & -3 \\
	3 & 5 \\
	-1 & 7
\end{pmatrix*} \begin{vect} x_1 \\ x_2 \end{vect} = \begin{vect} 3 \\ 2 \\ -5 \end{vect} \to \left(\!\begin{array}{rr|r}
	1 & -3 & 3 \\
	3 & 5 & 2 \\
	-1 & 7 & -5
\end{array}\!\right) \sim \left(\!\begin{array}{rr|r}
	1 & 0 & 1\sfrac{1}{2} \\
	0 & 1 & -\sfrac{1}{2} \\
	0 & 0 & 0
\end{array}\!\right) \to x_1 = 1\sfrac{1}{2}, x_2 = -\sfrac{1}{2} \]
\[ \vec{x} \in \mathbb{R}^2 \to x_3 \mbox{''vervalt''}) \to \vec{x} = \begin{vect} 1 \sfrac{1}{2} \\ - \sfrac{1}{2} \end{vect} \]
\end{enumerate}

\subsection{Meer definitie}
\[T: A \subset \mathbb{R}^n \to B \subset \mathbb{R}^m \]
Hierbij is $A$ het bereik en $B$ het bereik van de transformatie. $T$ is lineair als:
\begin{enumerate}
	\item $T(\vec{u} + \vec{v}) = T(\vec{u}) + T(\vec{v})$
	\item $T(c \cdot \vec{u}) = c \cdot T(\vec{u})$
\end{enumerate}

\paragraph{Voorbeeld} \emph{Afschuiving} (\emph{shear}) \index{afschuiving} \index{shear} van het vierkant gedefinieerd door de punten $(0,0) (2,0) (0,2) (2,2)$.
\[ T(\vec{x}) = A \vec{x}, \mathbb{R}^2 \to \mathbb{R}^2, A = \left(\!\begin{array}{rr}
	1 & 3 \\
	0 & 1
\end{array}\!\right) \]
\[ A\begin{vect} 2 \\ 0 \end{vect} = \begin{pmatrix*}[r]
	1 & 3 \\
	0 & 1
\end{pmatrix*} \begin{vect} 2 \\ 0 \end{vect} = \begin{vect} 2 \\ 0 \end{vect} \]
\[ A\begin{vect} 0 \\ 2 \end{vect} = \begin{pmatrix*}[r]
	1 & 3 \\
	0 & 1
\end{pmatrix*} \begin{vect} 0 \\ 2 \end{vect} = \begin{vect} 6 \\ 2 \end{vect} \]
\[ A\begin{vect} 2 \\ 2 \end{vect} = \begin{pmatrix*}[r]
	1 & 3 \\
	0 & 1
\end{pmatrix*} \begin{vect} 2 \\ 2 \end{vect} = \begin{vect} 8 \\ 2 \end{vect} \]

Deze transformaties zijn te zien in \autoref{fig:shear}. Hierbij zijn de blauwe vectoren en het blauwe vierkant het orgineel en de rode vectoren en het rode parallellogram zijn het beeld na de transformatie.

\begin{figure}[h!]
	\centering
	\begin{tikzpicture}
		\begin{axis}[xlabel=$x_1$, ylabel=$x_2$, axis x line=middle, axis y line=middle, xmin=-1, ymin=-1, xmax=9, ymax=3]

		\addplot[surf,mesh/rows=2, patch type=rectangle, blue, opacity=0.2] coordinates {(0,0) (2,0) (0,2) (2,2)};
		
		\addplot[surf,mesh/rows=2, patch type=rectangle, red, opacity=0.2] coordinates {(0,0) (2,0) (6,2) (8,2)};
		
		% (2, 0)
		\addplot [blue,-triangle 90] coordinates { (0,0.03) (2,0.03) };
		\node[coordinate, pin=300:{$\begin{vect} 2 \\ 0 \end{vect}$}] at (axis cs:2,0) {};
		
		% (0, 2)
		\addplot [blue,-triangle 90] coordinates { (0,0) (0,2) };
		\node[coordinate, pin=45:{$\begin{vect} 0 \\ 2 \end{vect}$}] at (axis cs:0,2) {};

		% (2, 2)
		\addplot [blue,-triangle 90] coordinates { (0,0) (2,2) };
		\node[coordinate, pin=45:{$\begin{vect} 2 \\ 2 \end{vect}$}] at (axis cs:2,2) {};
		
		% (2, 0)
		\addplot [red,-triangle 90] coordinates { (0,-0.03) (2,-0.03) };
		
		% (6, 2)
		\addplot [red,-triangle 90] coordinates { (0,0) (6,2) };
		\node[coordinate, pin=135:{$\begin{vect} 6 \\ 2 \end{vect}$}] at (axis cs:6,2) {};
		
		% (8, 2)
		\addplot [red,-triangle 90] coordinates { (0,0) (8,2) };
		\node[coordinate, pin=135:{$\begin{vect} 8 \\ 2 \end{vect}$}] at (axis cs:8,2) {};
		\end{axis}
	\end{tikzpicture}
	\caption{Plot van het aanpassen van het vierkant}
	\label{fig:shear}
\end{figure}

\paragraph{Voorbeeld} Wat is de afbeelding als je transformeert met $A = \begin{pmatrix*}[r]
	0 & -1 \\
	1 & 0
\end{pmatrix*}$?
\[ A \begin{vect} 4 \\ 1 \end{vect} = \begin{pmatrix*}[r]
	0 & -1 \\
	1 & 0
\end{pmatrix*} \begin{vect} 4 \\ 1 \end{vect} = \begin{vect} -1 \\ 4 \end{vect} \]
\[ A \begin{vect} 2 \\ 3 \end{vect} = \begin{pmatrix*}[r]
	0 & -1 \\
	1 & 0
\end{pmatrix*} \begin{vect} 2 \\ 3 \end{vect} = \begin{vect} -3 \\ 2 \end{vect} \]

Deze transformaties zijn afgebeeld in \autoref{fig:rotatie}. Hier valt te zien dat de vectoren $\sfrac{1}{2}\pi$ gedraaid zijn.

\begin{figure}[h!]
	\centering
	\begin{tikzpicture}
		\begin{axis}[xlabel=$x_1$, ylabel=$x_2$, axis x line=middle, axis y line=middle, xmin=-4, ymin=0, xmax=5, ymax=5]
		
		% (4, 1)
		\addplot [blue,-triangle 90] coordinates { (0,0) (4,1) };
		\node[coordinate, pin=90:{$\begin{vect} 4 \\ 1 \end{vect}$}] at (axis cs:4,1) {};
		% (2, 3)
		\addplot [blue,-triangle 90] coordinates { (0,0) (2,3) };
		\node[coordinate, pin=0:{$\begin{vect} 2 \\ 3 \end{vect}$}] at (axis cs:2,3) {};
		
		% (-1, 4)
		\addplot [red,-triangle 90] coordinates { (0,0) (-1,4) };
		\node[coordinate, pin=180:{$\begin{vect} -1 \\ 4 \end{vect}$}] at (axis cs:-1,4) {};
		% (-3, 2)
		\addplot [red,-triangle 90] coordinates { (0,0) (-3,2) };
		\node[coordinate, pin=90:{$\begin{vect} -3 \\ 2 \end{vect}$}] at (axis cs:-3,2) {};

		\end{axis}
	\end{tikzpicture}
	\caption{Plot van de rotatie}
	\label{fig:rotatie}
\end{figure}

\subsection{Transformatie eenheidsvectoren}
Als het bereik gegeven is maar de matrix $A$ niet dan moet je gebruik maken van de eenheidsvectoren $(\vec{e}_1, \ldots, \vec{e}_n)$ om de matrix $A$ te bepalen.

\paragraph{Voorbeeld} $T\left(\!\begin{vect} 1 \\ 0 \end{vect}\!\right) = \begin{vect} 5 \\ -7 \\ 2 \end{vect}, T\left(\!\begin{vect} 0 \\ 1 \end{vect}\!\right) = \begin{vect} -3 \\ 8 \\ 0 \end{vect}$ Vraag: waarheen leidt $\begin{vect} 2 \\ 1 \end{vect}$?

\[ T \begin{vect} x \\ y \end{vect} = T \left(\!x \begin{vect} 1 \\ 0 \end{vect} + y \begin{vect} 0 \\ 1 \end{vect} \!\right) = T \left(\! x \begin{vect} 1 \\ 0 \end{vect} \!\right) + T \left(\! y \begin{vect} 0 \\ 1 \end{vect} \!\right) = x T \begin{vect} 1 \\ 0 \end{vect} + y T \begin{vect} 0 \\ 1 \end{vect} \]
\[ = x \begin{vect} 5 \\ -7 \\ 2 \end{vect} + y \begin{vect} -3 \\ 8 \\ 0 \end{vect} = \begin{pmatrix*}[r]
	5 & -3 \\
	-7 & 8 \\
	2 & 0
\end{pmatrix*} \begin{vect} x \\ y \end{vect} \to A = \begin{pmatrix*}[r]
	5 & -3 \\
	-7 & 8 \\
	2 & 0
\end{pmatrix*} \]
\[T \begin{vect} 2 \\ 1 \end{vect} = A \begin{vect} 2 \\ 1 \end{vect} = \begin{pmatrix*}[r]
	5 & -3 \\
	-7 & 8 \\
	2 & 0
\end{pmatrix*} \begin{vect} 2 \\ 1 \end{vect} = \begin{vect} 7 \\ -6 \\ 4 \end{vect} \]

\paragraph{Voorbeeld} Wat is de matrix die hoort bij rotatie over de hoek $\phi$?
\[ \vec{e}_1 = \begin{vect} 1 \\ 0 \end{vect} \to \vec{p} = \begin{vect} \cos \varphi \\ \sin \varphi \end{vect}, \vec{e}_2 = \begin{vect} 0 \\ 1 \end{vect} \to \vec{q} = \begin{vect} - \sin \varphi \\ \cos \varphi \end{vect} \Rightarrow A = \begin{pmatrix*}[r]
	\cos \varphi & - \sin \varphi \\
	\sin \varphi & \cos \varphi
\end{pmatrix*} \]
Als $\varphi = \sfrac{1}{2} \pi \to A = \begin{pmatrix*}[r]
	0 & -1 \\
	1 & 0
\end{pmatrix*}$

\subsection{Eigenschappen van afbeeldingen}
\subsubsection{Definitie: injectie}
$T$ is injectief (''one-to-one'') als $T(\vec{x}) = T(\vec{y}) \to \vec{x} = \vec{y}$ ofwel als $\vec{x} \neq \vec{y} \to T(\vec{x}) \neq T(\vec{y})$. Verschillende originelen hebben verschillende afbeeldingen.

\paragraph{Voorbeeld} Is $T$ injectief? Altijd bij $T(\vec{0}) = \vec{0}$\[ T \begin{vect} 1 \\ 3 \\ -5 \end{vect} = \begin{pmatrix*}[r]
	2 & 1 & 1 \\
	3 & -1 & 0 \\
	5 & 0 & 1
\end{pmatrix*} \begin{vect} 1 \\ 3 \\ -5 \end{vect} = \begin{vect} 0 \\ 0 \\ 0 \end{vect} = \vec{0} \]
$\vec{0}$ heeft minstens 2 originelen dus $T$ is geen injectie.

\subsubsection{Definitie: surjectie}
$T$ is surjectief (''onto'') als $\forall b \in B \exists a \in A$ zodat $T(a) = b$. Ofwel: elk beeld heeft minstens \'e\'en origineel.

\paragraph{Voorbeeld} Is $T$ Surjectief?
\[ \left(\!\begin{array}{rrr|r}
	2 & 1& 1 & 1 \\
	3 & -1 & 0 & 1 \\
	5 & 0 & 1 & 1
\end{array}\!\right) \sim \left(\!\begin{array}{rrr|r}
	2 & 1 & 1 & 1 \\
	3 & -1 & 0 & 1 \\
	0 & 0 & 0 & -1
\end{array} \!\right) \]
Geen oplossingen dus er is geen orgineel. $T$ is geen surjectie.

\subsubsection{Definitie: bi-jectie}
$T$ is bi-jectief als $T$ zowel injectief als surjectief is.
\begin{eqnarray*}
T: A \to B &\mbox{ bi-jectief }& \forall b \in B \exists ! a \in A: T(a) = b \\
&& \forall a \in A \exists ! b \in B: T(a) = b
\end{eqnarray*}

\subsubsection{Bepalen injectie/surjectie/bi-jectie}
Om te bepalen of de transformatie $T$ een injectie, surjectie of bi-jectie is gebruik je de volgende twee stellingen:

\paragraph{Stelling 11:} \index{Stellingen!Hoofdstuk 1!Stelling 11} Stel $T: \mathbb{R}^n \to \mathbb{R}^m$. Dan is $T$ een injectie dan en slechts dan als de vergelijking $T(\vec{x}) = \vec{0}$ alleen de nulvector als oplossing heeft.

\paragraph{Stelling 12:} \index{Stellingen!Hoofdstuk 1!Stelling 12} Stel $T: \mathbb{R}^n \to \mathbb{R}^m$ en $A$ is de bijbehorende matrix, dan:
\begin{enumerate}
	\item $T$ is een surjectie dan en slechts dan als alle vectoren is $\mathbb{R}^m$ een lineaire combinatie zijn van de kolommen van $A$.
	\item $T$ is een injectie dan en slechts dan als de kolommen van $A$ lineair onafhankelijk zijn.
\end{enumerate}

Het is ook zo dat als de transformatie matrix $A$ inverteerbaar is dan is de transformatie $T$ een bi-jectie.


\subsection{Handige standaard transformaties}
\begin{table}[h]
	\centering
	\begin{tabular}{l|c}
		\textbf{Transformatie} & \textbf{Standaard matrix} \\ \hline
		Spiegelen door de $x_1$-as & $\begin{pmatrix*}[r] 1 & 0 \\ 0 & -1 \end{pmatrix*}$ \\ \hline
		Spiegelen door de $x_2$-as & $\begin{pmatrix*}[r] -1 & 0 \\ 0 & 1 \end{pmatrix*}$ \\ \hline
		Spiegelen door de lijn $x_2 = x_1$ & $\begin{pmatrix} 0 & 1 \\ 1 & 0 \end{pmatrix}$ \\ \hline
		Spiegelen door de lijn $x_2 = -x_1$ & $\begin{pmatrix*}[r] 0 & -1 \\ -1 & 0 \end{pmatrix*}$ \\ \hline
		Spiegelen door de oorsprong & $\begin{pmatrix*}[r] -1 & 0 \\ 0 & -1 \end{pmatrix*}$ \\ \hline
		Horizontaal rekken/krimpen met factor $k$ & $\begin{pmatrix} k & 0 \\ 0 & 1 \end{pmatrix}$ \\ \hline
		Verticaal rekken/krimpen met factor $k$ & $\begin{pmatrix} 1 & 0 \\ 0 & k \end{pmatrix}$ \\ \hline
		Horizontale verschuiving met factor $k$ & $\begin{pmatrix} 1 & k \\ 0 & 1 \end{pmatrix}$ \\ \hline
		Verticaal verschuiven met factor $k$ & $\begin{pmatrix} 1 & 0 \\ k & 1 \end{pmatrix}$ \\ \hline
		Projectie op de $x_1$-as & $\begin{pmatrix} 1 & 0 \\ 0 & 0 \end{pmatrix}$ \\ \hline
		Projectie op de $x_2$-as & $\begin{pmatrix} 0 & 0 \\ 0 & 1 \end{pmatrix}$ \\  \hline
		Rotatie om de oorsprong (tegen de klok in) & $\begin{pmatrix*}[r] \cos \varphi & - \sin \varphi \\	\sin \varphi & \cos \varphi \end{pmatrix*}$
	\end{tabular}
	\caption{Overzicht van standaard transformaties}
	\label{tbl:standtrans}
\end{table}