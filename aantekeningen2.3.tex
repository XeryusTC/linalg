\section{Eigenschappen van $A^{-1}$ (\S2.3)}
\subsection{Stelling 8: Inverteerbare matrix theorem} \index{Stellingen!Hoofdstuk 2!Stelling 8} \index{Inverteerbare matrix stelling} 12 gelijkwaardige uitspraken, als er \'e\'en geldt dan gelden ze allemaal. Het omgekeerde is ook waar, als \'e\'en niet waar is dan gelden ze allemaal niet. $A$ is hierbij een matrix van $(n \times n)$ grootte.
\begin{enumerate}
	\item $A$ is inverteerbaar.
	\item $A$ is rij equivalent aan de $(n \times n)$ eenheidsmatrix.
	\item $A$ heeft $n$ pivots.
	\item $A \vec{x} = \vec{0}$ heeft alleen $\vec{x} = \vec{0}$ als oplossing.
	\item De kolommen van $A$ zijn lineair onafhankelijk.
	\item De lineaire transformatie van $A \vec{x}$ is een injectie.
	\item $A \vec{x} = \vec{b}$ heeft op zijn minst een oplossing voor elke $\vec{b}$ in $\mathbb{R}^n$.
	\item De kolommen van $A$ spannen $\mathbb{R}^n$ op.
	\item De lineaire transformatie van $A \vec{x}$ is surjectief voor $\mathbb{R}^n$ naar $\mathbb{R}^n$.
	\item Er is een $(n \times n)$ matrix $C$ zodat $CA = I$.
	\item Er is een $(n \times n)$ matrix $D$ zodat $AD = I$.
	\item $A^T$ is inverteerbaar.
	\item De determinant van $A \neq 0$. (Stelling 4) \index{Stellingen!Hoofdstuk 3!Stelling 4}
\end{enumerate}

\paragraph{Voorbeeld}
$A = \begin{pmatrix*} 1 & 0 & -2 \\
3 & 1 & -2 \\
-5 & -1 & 9 \end{pmatrix*}$ Bestaat $A^{-1}$?
Kijk naar 4. $A \vec{x} = \vec{0}$ all\'e\'en als $\vec{x} = \vec{0}$. $A \vec{x} = \vec{0}$ geeft:
\[ \left(\!\begin{array}{rrr|r} 1 & 0 & -2 & 0 \\
3 & 1 & -2 & 0 \\
-5 & -1 & 9 & 0 \end{array}\!\right) \sim \left(\!\begin{array}{rrr|r} 1 & 0 & -1 & 0 \\
0 & 1 & 4 & 0 \\
0 & 0 & 1 & 0 \end{array}\!\right) \to x_3 = 0 \to x_2 = 0 \to x_1 = 0\]
$\vec{x} = \vec{0}$ is dus de enige oplossing dus $A^{-1}$ bestaat.

\subsection{Toepassing van $A^{-1}$}
\begin{eqnarray*}
	T: \vec{x} &\to& A \vec{x} \\
	&\to& A^{-1}A \vec{x} = \vec{x}
\end{eqnarray*}
\[ \left. \begin{array}{ll} T: & \vec{x} \to A \vec{x} \\
S: & \vec{x} \to A^{-1} \vec{x} \end{array}\right\} \mbox{Zijn elkaars omgekeerde (alleen als} A^{-1} \mbox{bestaat}) \]
$T$ en $S$ zijn daarom allebei bi-jectief. Het is dus mogelijk om aan te tonen dat $T$ bi-jectief is door aan te tonen dat $A^{-1}$ bestaat.